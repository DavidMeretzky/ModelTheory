\documentclass{article}

\usepackage{algorithmic, amsmath, amsthm, amsfonts, amssymb, enumerate, tikz, tikz-cd, color, mathrsfs} %tikz is for drawing lattices %tikz-cd is for commutative diagrams
															%color is for making notes in red 
															%mathrsfs is for power set font
%\usepackage[mathscr]{eucal} %mathscr gives nice script fonts

\newtheoremstyle{problemstyle}  % <name> This is my problemstyle. use begin{problem}.
        {12pt}                                               % <space above>
        {}                                               % <space below>
        {\itshape}                               % <body font>
        {}                                                  % <indent amount}
        {\bfseries}                 % <theorem head font>
        {\normalfont\bfseries.}         % <punctuation after theorem head>
        {.5em}                                          % <space after theorem head>
        {}                                                  % <theorem head spec (can be left empty, meaning `normal')>


\theoremstyle{problemstyle}

\newtheorem{problem}{Problem}


\title{ \vspace{-10ex} %uncomment to remove vertical space
%title of assignment goes here e.g. "Math 721 Homework 3"
An Algebraic Introduction to Mathematical Logic\\
Chapter 1 Universal Algebra \\
Section 1 Introduction \\
Exercises 
}


\author{David L. Meretzky
}


\date{%date assignment is due goes here
February 1st, 2018
} 


\renewcommand*{\thefootnote}{$\dagger$} %changes default footnote marking to a dagger instead of a number (numbers are sometimes mistaken for citations)

\begin{document}

\maketitle

\begin{flushleft}
$\textbf{Preliminary Definition of Operation:}$
An $n$-ary operation on the set $A$ is a function $t:A^{n}\rightarrow A$. The number $n$ is called the arity of $t$.
\end{flushleft}
\begin{flushleft}
$\textbf{Ex. 1.3}$ A $0$-ary operation on a set $A$ is a function from the set $A^0$ (whose only element is $\emptyset$)
\end{flushleft}
\begin{flushleft}
$\textbf{Definition 1.4}$ A $type$ $\mathscr{T}$ is a set $T$ together with a function $ar:T\rightarrow\mathbb{N}$. We shall write, $\mathscr{T} = (T,ar)$, or, more simply, abuse notation and denote the type by $T$. It is also convenient to denote by $T_n$ the set $\{t \in T|ar(t) = n\}.$
\end{flushleft}
\begin{flushleft}
$\textbf{Definition 1.5}$ An $algebra$ of type $T$, or a $T$-$algebra$, is a set $A$ together with, for each $t \in T$, a function $t_A:A^{ar(t)}\rightarrow A$.  The elements $t \in T_n$ are called $n$-ary $T$-algebra operations. 
\end{flushleft}

\begin{flushleft}
$\textbf{Definition 1.11}$ If $A$ is a $T$-algebra, then a subset $B \subset A$ is called a $T$-$subalgebra$ of $A$ if it forms a $T$-algebra with operations the restrictions to $B$ of those on $A$. That is to say, for each n-ary operation $t \in T_{n}$ on A, restricting the domain of $t$ to just the set $B$, we have that $t|_B$ is an n-ary operation on B. For all $b_1, ... b_n \in B$, we have $t(b_1, ... b_n)|_B  = t(b_1, ... b_n) \in B$
\end{flushleft}

\begin{problem}[1.12 a] 
$A$ is a $T$-algebra. Show that $\emptyset$ is a subalgebra if and only if $T_{0} = \emptyset$. 
\end{problem}

\begin{proof}
The empty set is clearly contained in $A$. For any $n > 0$,
$t_n: A^n \rightarrow A$ \\Since no element of $A^n$ could possibly be contained in the empty set, 
the image of $\emptyset$ under $T_n$, for all operations of arity $n >0$, must be the empty set. Thus we have shown for all $T_n|_{\emptyset}(\emptyset) = \emptyset$ where $n > 0$.  It remains to show what happens in the case $T_0$.

If $T_0$ sends $\emptyset$ to $\emptyset$, then under all operations, $T_n$, the image of $\emptyset$ is itself. Thus if $T_{0} = \emptyset$ then $\emptyset$ is a subalgebra.

Suppose $\emptyset$ constitutes a $T$-$subalgebra$ for the $T$-algebra $A$. Then all operations of $A$ send $\emptyset$ to itself. And therefore, the arity $0$ operation $T_0$ sends $\emptyset$ to itself. 
\end{proof}

\begin{flushleft}
Proposition: Any intersection of subalgebras is a subalgebra. Given any subset $X \subset A$, there is a unique smallest subalgebra containing $X$-namely, the subalgebra $\bigcap\{U|$ $U$ $subalgebra$ $of$ $A$ $and$ $X\subseteq U\}$. we call this the subalgebra generated by $X$ and denote it $<X>_T$, or if there is no risk of confusion $<X>$. 
\end{flushleft}

\begin{problem}[1.12 b] 
Show that for all $T$, every $T$-algebra has a unique smallest subalgebra. 
\end{problem}
\begin{proof}[Solution (1.12 b)] 
I claim that the unique smallest subalgebra of any $T$-$algebra$ A is the subalgebra generated by $\emptyset$, $<\emptyset>_T$. Let $U$ be any subalgebra of $A$, $\emptyset \in U$. Therefore, 

$<\emptyset>_T$ =  \\$\bigcap\{U|$ $U$ $is$ $a$ $subalgebra$ $of$ $A$ $and$ $\emptyset \subseteq U\} =$ $\bigcap\{U|$ $U$ $is$ $a$ $subalgebra$ $of$ $A\}$  

All subalgebras $U$ of $A$, appear in this intersection, therefore $<\emptyset>_T \subseteq U$ for all $U$. Uniqueness follows because $<\emptyset>_T$ is a subalgebra of $A$, and therefore if $V$ is any other subalgebra with the property of being the smallest, it would have to be contained in $<\emptyset>_T$. So $V \subseteq <\emptyset>_T$ and $<\emptyset>_T \subseteq V$, so $V  = <\emptyset>_T$
\end{proof}

\begin{problem}[1.13 a]
Groups may be regarded as the special case of $T$-algebras where $T = (\{*\},ar)$
 with $ar(*) = 2$, or of $T'$-algebras where $T' = (\{e,i,*\},ar)$, $ar(e) = 0$, $ar(i) = 1$, and $ar(*) = 2$.  
 
Show that every $T'$-subalgebra of a group is a subgroup but not every non-empty $T$-subalgebra need be a group.
\end{problem}
\begin{proof}[1.13 a]
Let $G$ be a $T'$-algebra and let $H$ be any subalgebra. Then since $H$ contains the empty set, $T'_n(\emptyset) = T'_n|_H(\emptyset) = e \in H$. So $H$ has an identity. The fact that it is a subalgebra also means that it is closed under $i$ and $*$. 

Let $g$ be an element of a group $G$ such that $g$ has infinite order. Letting $H$ be the set of all products of $g$ we see that $g$ is a subalgebra. For any $m,n \geq 1$, we obtain $g^n * g^m = g^{m+n} \in H$. $H$ is easily seen to not include the identity because otherwise $\exists k$ s.t. $g^k = e$. Therefore, $H$ is not a subgroup. 
\end{proof}
\begin{problem}[1.13 b]
Show that if $G$ is a finite group, then every non-empty $T$-subalgebra of $G$ is itself a group.
\end{problem}
\begin{proof}[1.13 b]
This is a consequence of Lagrange's Theorem. 
\end{proof}
\end{document}


