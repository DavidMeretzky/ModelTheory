\documentclass{article}

\usepackage{algorithmic, amsmath, amsthm, amsfonts, amssymb,commath, enumerate, tikz, tikz-cd, color, mathrsfs} %tikz is for drawing lattices %tikz-cd is for commutative diagrams
															%color is for making notes in red 
															%mathrsfs is for power set font
%\usepackage[mathscr]{eucal} %mathscr gives nice script fonts

\newtheoremstyle{problemstyle}  % <name> This is my problemstyle. use begin{problem}.
        {12pt}                                               % <space above>
        {}                                               % <space below>
        {\itshape}                               % <body font>
        {}                                                  % <indent amount}
        {\bfseries}                 % <theorem head font>
        {\normalfont\bfseries.}         % <punctuation after theorem head>
        {.5em}                                          % <space after theorem head>
        {}                                                  % <theorem head spec (can be left empty, meaning `normal')>


\theoremstyle{problemstyle}

\newtheorem{problem}{Problem}


\title{ \vspace{-10ex} %uncomment to remove vertical space
%title of assignment goes here e.g. "Math 721 Homework 3"
An Algebraic Introduction to Mathematical Logic\\
Chapter 2 Propositional Calculus \\
Section 4 Proof in The Propositional Calculus \\
Exercises 
}


\author{David L. Meretzky
}


\date{%date assignment is due goes here
February 7th, 2018
} 


\renewcommand*{\thefootnote}{$\dagger$} %changes default footnote marking to a dagger instead of a number (numbers are sometimes mistaken for citations)

\begin{document}

\maketitle

\begin{flushleft}
For the propositional calculus on the set $X$, we take as axioms all elements of the subset $\mathscr{A} = \mathscr{A}_1 \cup \mathscr{A}_2 \cup \mathscr{A}_3$ 
of $P(X)$, where 
\end{flushleft}

\begin{flushleft}
$\mathscr{A}_1 = \{p \Rightarrow (q \Rightarrow p)|p,q \in P(X) \}$ 

$\mathscr{A}_2 = \{(p \Rightarrow (q \Rightarrow r)) \Rightarrow ((p \Rightarrow q)\Rightarrow (p \Rightarrow r))|p,q,r \in P(X) \}$ and 

$\mathscr{A}_3 = \{\sim$$\sim$$p\Rightarrow p|p \in P(X) \}$.
\end{flushleft}

\begin{flushleft}
As our one rule of inference, we take the rule known as modus ponens: from $p$ and $p\Rightarrow q$, deduce $q$. We may now give a formal definition of a proof. 
\end{flushleft}

\begin{flushleft}
\textbf{Definition 4.1}
Let $q \in P(X)$ and let $A \subseteq P(X)$. In the propositional calculus on the set $X$, a $\textit{proof of q from the assumptions A}$ is a finite sequence $p_1, p_2,$ ... $p_n$ of elements $p_i \in P(X)$ such that $p_n = q$ and for each $i$, either $p_i \in \mathscr{A} \cup A$ or for some $j,k < i$ we have $p_k = (p_j \Rightarrow p_i)$.
\end{flushleft}
\begin{flushleft}
\textbf{Definition 4.2}
Let $q \in P(X)$ and let $A \subseteq P(X)$. We say that $q$ is a $\textit{deduction}$ from $A$, or $q$ is $provable$ from $A$, or that $A$ $syntactically$ $implies$ $q$, if there exists a proof of $q$ from $A$. We shall write this as $\vdash$, and we shall denote by $Ded(A)$ the set of all deduction from $A$.  
\end{flushleft}

\begin{problem}[4.9] 
Show that $Ded(A)$ is the smallest subset $D$ of $P(X)$ such that $\mathscr{A} \cup A \subseteq D$ and closed under modus ponens. 
\end{problem}

\begin{proof}[4.9 Solution]
Pick any $r \in Ded(A)$. We will show that it must be in any other subset $D$ of $P(X)$ satisfying the above requirements. This is sufficient to show that $Ded(A) \subseteq D$. 

Since $r \in Ded(A)$, there exists a proof of $r$. If the proof is of length $1$, then $r \in \mathscr{A} \cup A$ and therefore in $D$, and we are finished. The result follows inductively, suppose it holds that $r \in D$ for all $r$ with proofs of up to length $n-1$. Then either $p_n = r \in \mathscr{A} \cup A$ or for some $i,j<n$ $p_i = p_j \Rightarrow r$ in which case since $p_i$ and $p_j$ have proofs of length $n-1$ or less, $p_j$ and $p_j$ lie in $D$.  By closure under modus ponens, $r \in D$. 

Therefore, $Ded(A) \subseteq D$ for all such $D$ and is thus the smallest, ordered by inclusions. 
\end{proof}

\begin{problem}[4.10] 
Construct a proof in the propositional calculus of $p \Rightarrow r$ from the assumptions $A = \{p\Rightarrow q, q\Rightarrow r\}$. 
\end{problem}
\begin{proof}[4.10] 
By assumption, $(q \Rightarrow r) \in Ded(A)$. Let $p_1 = (q \Rightarrow r)$. Axiom $1$ of $\mathscr{A}$ says that $p_1 \Rightarrow (p \Rightarrow p_1) \in \mathscr{A} \cup A$ and is therefore in $Ded(A)$.\\\\We have, $p_1 \in Ded(A)$ and $p_1 \Rightarrow (p \Rightarrow p_1) \in Ded(A)$. By modus ponens, $p \Rightarrow p_1 \in Ded(A)$. Applying Axiom $2$ to $(p \Rightarrow p_1) = (p \Rightarrow (q \Rightarrow r))$, We obtain $((p \rightarrow q) \Rightarrow (p \Rightarrow r))$. Noting that we began with the assumption $(p \rightarrow q)$, by modus ponens we obtain $(p \Rightarrow r)$.
\end{proof}



\end{document}


