\documentclass{article}

\usepackage{algorithmic, amsmath, amsthm, amsfonts, amssymb,commath, enumerate, tikz, tikz-cd, color, mathrsfs} %tikz is for drawing lattices %tikz-cd is for commutative diagrams
															%color is for making notes in red 
															%mathrsfs is for power set font
%\usepackage[mathscr]{eucal} %mathscr gives nice script fonts

\newtheoremstyle{problemstyle}  % <name> This is my problemstyle. use begin{problem}.
        {12pt}                                               % <space above>
        {}                                               % <space below>
        {\itshape}                               % <body font>
        {}                                                  % <indent amount}
        {\bfseries}                 % <theorem head font>
        {\normalfont\bfseries.}         % <punctuation after theorem head>
        {.5em}                                          % <space after theorem head>
        {}                                                  % <theorem head spec (can be left empty, meaning `normal')>


\theoremstyle{problemstyle}

\newtheorem{problem}{Problem}


\title{ \vspace{-10ex} %uncomment to remove vertical space
%title of assignment goes here e.g. "Math 721 Homework 3"
An Algebraic Introduction to Mathematical Logic\\
Chapter 2 Propositional Calculus \\
Section 3 Truth in Propositional Calculus \\
Exercises 
}


\author{David L. Meretzky
}


\date{%date assignment is due goes here
February 4th, 2018
} 


\renewcommand*{\thefootnote}{$\dagger$} %changes default footnote marking to a dagger instead of a number (numbers are sometimes mistaken for citations)

\begin{document}

\maketitle

\begin{flushleft}
\textbf{Definition 3.1}
A $valuation$ of P(X) is a proposition algebra homomorphism $v:P(X) \rightarrow \mathbb{Z}_2$. We say that $p \in P(X)$ is $\textit{true with respect to v}$ if $v(p) = 1$, and that $p$ is $\textit{false with respect to v}$ if $v(p) = 0.$
\end{flushleft}
\begin{flushleft}
\textbf{Definition 3.2}
Let $A\subseteq P(x)$ and $q \in P(x)$. We say that $q$ is a $consequence$ of the set $A$ of assumptions, or that $\textit{A semantically implies q}$, if $v(q) = 1$ for every valuation $v$ such that $v(p) = 1$ for all $p \in A$. We shall write this $A \models q$, and we shall denote by $Con(A)$, the set $\{p\in P(X)|A\models p\}$ of all consequences of $A$.
\end{flushleft}

\begin{problem}[3.6] 
Show that $\{F\}\models p$ for all $p \in P(x)$
\end{problem}

\begin{proof}[3.6 Solution]
$p = \sim\sim$$p$
Therefore, given any valuation $v$ with the property that $v(F) = 1$ we have $v(\sim\sim$$p) = v(1+\sim$$p(1+F))=v(1+\sim$$p(1+1)) = v(1+\sim$$p*0) = 1$. Therefore $\{F\}$ semantically implies $p$ for all $p \in P(X)$. 
\end{proof}

\begin{problem}[3.7 a] 
Show that $\{p, p\Rightarrow q\}\models q$ for all $p,q \in P(x)$.
\end{problem}
\begin{proof}[Solution 3.7 a] 
We have, $1 = v(1+p(1+q))$ and $v(p) = 1$. Therefore, since $v$ is a homomorphism, $1 = v(1+1(1+q))$ this means that $1 = 1 + v(1+q)$ so $0 = 1 + v(q)$ therefore we have $v(q) = 1$. 
\end{proof}

\begin{problem}[3.7 b] 
Show that $\{p, \sim$$q\Rightarrow \sim$$p\}\models q$ for all $p,q \in P(x)$
\end{problem}

\begin{proof}[Solution 3.7 b] 
We have, $1 = v(1 + \sim$$q(1+\sim$$p))$ and $v(p) = 1$. Therefore, $v(\sim$$p) = 0$. Therefore,$1 = v(1 + \sim$$q(1+\sim$$p)) = v(1+\sim$$q(1+0)) = v(1+\sim$$q)$. We finally obtain $0 = v(\sim$$q)$ which implies, $v(q) = 1$.
\end{proof}

\begin{problem}[3.8]
Show that $p \Rightarrow (p \Rightarrow q)$ is a tautology. 
\end{problem}
\begin{proof}[Solution 3.8]
$v(p \Rightarrow (p \Rightarrow q)) = v(1+ p(1+ p \Rightarrow q)) = v(1+ p(1+1+p(1+q))) = v(1+p(q(1+p)) = 1$ When $p = 0$ this expression values to $v(1+0(q(1+p))) = 1$. Similarly, when $p = 1$ this expression values to $v(1+1(q(1+1))) = v(1+(q*0)) = v(1) = 1$. Therefore, regardless of what $q$ is, this expression is always true. 
\end{proof}


\end{document}


