\documentclass{article}

\usepackage{algorithmic, amsmath, amsthm, amsfonts, amssymb,commath, enumerate, tikz, tikz-cd, color, mathrsfs,enumitem} %tikz is for drawing lattices %tikz-cd is for commutative diagrams
															%color is for making notes in red 
															%mathrsfs is for power set font
%\usepackage[mathscr]{eucal} %mathscr gives nice script fonts

\newtheoremstyle{problemstyle}  % <name> This is my problemstyle. use begin{problem}.
        {12pt}                                               % <space above>
        {}                                               % <space below>
        {}                               % <body font>
        {}                                                  % <indent amount}
        {\bfseries}                 % <theorem head font>
        {\normalfont\bfseries.}         % <punctuation after theorem head>
        {.5em}                                          % <space after theorem head>
        {}                                                  % <theorem head spec (can be left empty, meaning `normal')>


\theoremstyle{problemstyle}

\newtheorem{problem}{Problem}

\theoremstyle{lemmastyle}

\newtheorem{lemma}{Lemma}

\theoremstyle{theoremstyle}

\newtheorem{theorem}{Theorem}

\theoremstyle{problemstyle}

\newtheorem{definition}{Definition}

\theoremstyle{problemstyle}

\newtheorem{corollary}{Corollary}

\setlength\parindent{0pt}

\title{ \vspace{-10ex} %uncomment to remove vertical space
%title of assignment goes here e.g. "Math 721 Homework 3"
An Algebraic Introduction to Mathematical Logic\\
Chapter 4 Predicate Calculus \\
Section 4 Properties of $Pred(V, \mathscr{R})$ \\
}


\author{David L. Meretzky
}


\date{%date assignment is due goes here
December 30, 2018
} 


\renewcommand*{\thefootnote}{\roman{footnote}}%{$\dagger$} %changes default footnote marking to a dagger instead of a number (numbers are sometimes mistaken for citations)

\begin{document}

\maketitle

\subsection*{Some informal comments}

Recall that the logic called the First Order Predicate Calculus is comprised of an infinite set of variables $V$, a set of relation symbols $\mathscr{R}$, which together yeild a set of generators of the form $r(x_1,...,x_n)$, where $r \in \mathscr{R}$ and each $x_i \in V$ of a free $\{\textbf{F}, \Rightarrow, (\forall x) | x \in V\}$-algebra, called $\widetilde{P}(V,\mathscr{R})$. We call the factor algebra by the equivalence relation detailed in Chapter 4 Section 1 $P(V,\mathscr{R})$. When we equip $P(V,\mathscr{R})$ with semantics and syntax constructed in the previous two sections and denote the logic $Pred(V,\mathscr{R})$. 

\begin{definition}
A predicate $p$ is called a sentence if $var(p) = \emptyset$, that is, $p$ has no unbound variables. Otherwise we call $p$ a fragment. 
\end{definition}

Note that this corresponds to our intuitive notion. Not that the following is a sentence: ``For all deltas, there exists an epsilon." This is an honest to god sentence since both variables delta and epsilon are quantified over. The following is not strictly: ``For all clocks, rabbit." We can fix this as follows: ``For all clocks, a rabbit is late." The word ``a" now plays the role of the the existencial quantifier. \\

What is the proper notion of morphism for $\{\textbf{F}, \Rightarrow, (\forall x) | x \in V\}$-algebras? The reliance of the quantifiers on the set of variables $V$ implies that the variables of the two logics must be the same.\\

It turns out that homomorphism is too strict to be of use. Let $\phi$ be a homomorphism between two such algebras, $P_1$, and $P_2$. Let $p(x),q(y) \in P_1$, then $$\phi((\forall x)p(x) \Rightarrow q(x)) =\phi((\forall x)p(x)) \Rightarrow \phi(q(x)) =  ((\forall x)\phi(p(x))) \Rightarrow \phi(q(x)).$$ By definition, $\phi$ may not interact with quantifiers, $\textbf{F}$, or $\Rightarrow$. Thus the only parts of the predicate that it can interact with are variables (bound or unbound) and relation symbols.\\

On a high level, we would like morphisms to preserve the meaning of the predicate. Barnes and Mack have little to say on how relation symbols change.\\

I think there are probably some easy contradictions that one runs into if we allow homomorphisms to change the arity of a relation symbol.  

\begin{enumerate}
\item Can $\{\textbf{F}, \Rightarrow, (\forall x) | x \in V\}$-algebra homomorphisms alter relation symbols at all?
\item Must $\{\textbf{F}, \Rightarrow, (\forall x) | x \in V\}$-algebra homomorphisms preserve the arity of the relation symbol?
\end{enumerate}

Barnes and Mack have more to say regarding the behavior of homomorphisms with respect to variables. To be even close to interesting, homomorphisms must be able to interchange variables. Suppose $\phi$ sends $x$ to $y$ and fixes all relation symbols. Then if $x$ is bound in a predicate $p$, we run into issues. Computing, $$\phi((\forall x)p(x)) = (\forall x)\phi(p(x)) = (\forall x)p(y),$$ we see that the homomorphism sends a sentence to a fragment.\\

Barnes and Mack give the following definition which allows for greater flexibility. 

\subsection*{Properties of $Pred(V,\mathscr{R})$}

\begin{definition}
Let $P_1 = P(V_1,\mathscr{R}^{(1)})$ and $P_2 = P(V_2,\mathscr{R}^{(2)})$. A \textit{semi-homomorphism} $(\alpha,\beta):(P_1,V_1) \rightarrow (P_2,V_2)$ is a pair of maps $\alpha:P_1 \rightarrow P_2$, $\beta:V_1 \rightarrow V_2$ such that 
\begin{enumerate}
\item $\beta(V_1)$ is infinite. 
\item $\alpha$ is an $\{F, \Rightarrow\}$-algebra homomorphism, and 
\item $\alpha((\forall x)p) = (\forall x')\alpha(p)$ where $x' = \beta(x)$.  
\end{enumerate}
\end{definition}

\begin{lemma}
$var((\forall x)p) = var(p)$ iff $x \notin var(p)$. 
\end{lemma}

\begin{proof}
Suppose that $x \in var(p)$ then $var((\forall x)p) \neq var(p)$ since $x \notin var((\forall x)p)$ but $x \in var(p)$. Suppose now that $x \notin var(p)$. Then $var(p)-\{x\} = var(p)$. Since $var((\forall x)p) = var(p) - \{x\}$ the result follows. 
\end{proof}

\begin{lemma}
Suppose that $x \neq y$, then $(\forall x)p = (\forall y)p$ iff $x,y \notin var(p)$.
\end{lemma}

\begin{proof}
One direction follows immediately from lemma 1. The other is as follows which we prove using the contrapositive. Suppose $x,y \in var(p)$ then $var((\forall x)p) \neq var((\forall y)p)$ because $y \in var((\forall x)p)$ but $y \notin var((\forall y)p)$ and thus $(\forall x)p \neq (\forall y)p$. 
\end{proof}

\begin{lemma}
Let $(\alpha,\beta):(P_1,V_1) \rightarrow (P_2,V_2)$ be a semi-homomorphism. Let $p \in P_1$ and suppose that $x \in V_1-var(p)$. Then $\beta(x) \notin var(\alpha(p))$.  
\end{lemma}

\begin{proof}
Since $\beta(V_1)$ is infinite there exists a $y' \in \beta(V_1)$ such that $y' \notin \beta(var(p))$ and $y' \neq \beta(x)$. Since $y' \in \beta(V_1)$ there exists a $y \in V_1$ such that $\beta(y) = y'$. Additionally choose $y'$ so that $y \in V_1-var(p)$.  It follows from lemma 2 that $(\forall x)p = (\forall y)p$. Then we may compute $$(\forall \beta(x))p = \alpha((\forall x) p) = \alpha((\forall y) p)=(\forall \beta(y))p$$
From which we may conclude by lemma 3 that $\beta(x) \notin var(\alpha(p))$. 
\end{proof}


\begin{theorem}(The substitution theorem) Let $(\alpha,\beta):(P_1,V_1) \rightarrow (P_2,V_2)$ be a semi-homomorphism. Let $A \subseteq P_1$\footnote{the text has just  $A \subseteq P$ which must be a typo.}, $p \in P_1$. Then

\begin{enumerate}[label = (\alph*)]

\item If $A \vdash p$, then $\alpha(A) \vdash \alpha(p)$.
\item If $A \models p$, then $\alpha(A) \models \alpha(p)$.

\end{enumerate}
\end{theorem}

\begin{proof}

\end{proof}

\end{document}