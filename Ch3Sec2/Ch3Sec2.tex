\documentclass{article}

\usepackage{algorithmic, amsmath, amsthm, amsfonts, amssymb,commath, enumerate, tikz, tikz-cd, color, mathrsfs} %tikz is for drawing lattices %tikz-cd is for commutative diagrams
															%color is for making notes in red 
															%mathrsfs is for power set font
%\usepackage[mathscr]{eucal} %mathscr gives nice script fonts

\newtheoremstyle{problemstyle}  % <name> This is my problemstyle. use begin{problem}.
        {12pt}                                               % <space above>
        {}                                               % <space below>
        {\itshape}                               % <body font>
        {}                                                  % <indent amount}
        {\bfseries}                 % <theorem head font>
        {\normalfont\bfseries.}         % <punctuation after theorem head>
        {.5em}                                          % <space after theorem head>
        {}                                                  % <theorem head spec (can be left empty, meaning `normal')>


\theoremstyle{problemstyle}

\newtheorem{problem}{Problem}


\title{ \vspace{-10ex} %uncomment to remove vertical space
%title of assignment goes here e.g. "Math 721 Homework 3"
An Algebraic Introduction to Mathematical Logic\\
Chapter 3 Propositional Calculus \\
Section 2 Soundness and Adequacy of Prop(X) \\
Exercises 
}


\author{David L. Meretzky
}


\date{%date assignment is due goes here
February 9th, 2018
} 


\renewcommand*{\thefootnote}{$\dagger$} %changes default footnote marking to a dagger instead of a number (numbers are sometimes mistaken for citations)

\begin{document}

\maketitle

\begin{flushleft}
\textbf{Theorem 2.1} (The Soundness Theorem) Let $A \subseteq P(X)$, $p \in P(X)$. $If$ $A \vdash p$, $then$ $A \models p$.
\end{flushleft}

\begin{flushleft}
\textbf{Corollary 2.2} (The Consistency Theorem) $F$ $\textit{is not a theorem of}$ $Prop(x)$.
\end{flushleft}

\begin{problem}[Exercise 2.3] 
Show that $Con(A)$ is closed with respect to modus ponens (i.e. if $p$, $p \Rightarrow q \in Con(A)$, then $q \in Con(A)$). Use Exercise 4.9 of Chapter II to prove that $Ded(A) \subseteq Con(A)$. This is another way of stating the soundness theorem. 
\end{problem}

\begin{proof}[Exercise 2.3]
We show first that $Con(A)$ is closed under modus ponens. Let $A\models p$ and $A \models p\Rightarrow q$. Then for all valuations $v$ such that $v(A) = 1$, we have $v(p)=1$ and $v(p\Rightarrow q) = 1$. \\\\
$v(p \Rightarrow q) = v(1 + p(1 + q))$\\\\
Since we have that $v$ is a homomorphism,\\\\
$v(1 + p(1 + q)) = 1 + v(p)(1 + v(q)) = 1 + 1*(1 + v(q)) = 1 + (1 + v(q))$,\\\\
Since we are working in $\mathbb{Z}_2$\\\\
$v(q)+1 = 0$ which implies $v(q) = 1$. \\

This completes the verification that $A\models q$, and hence $A$ is closed under modus ponens. 

By exercise 4.9 of Chapter II Section IV, we have that $Ded(A)$ is the smallest subset of $P(X)$ which is closed under modus ponens and contains all of the axioms. Since the axioms are tautologies, (valid for all valuations), we must have $Ded(A) \subseteq Con(A)$.
\end{proof}

\begin{flushleft}
\textbf{Theorem 2.4} (The Deduction Theorem) Let $A \subseteq P(X)$, and let $p,q \in P(X)$. Then $A\vdash p\Rightarrow q$ $if$ $and$ $only$ $if$ $A \cup \{p\}\vdash q$.
\end{flushleft}

\begin{problem}[Exercise 2.6]
Show that $p \Rightarrow r \in Ded\{p\Rightarrow q, p \Rightarrow (q\Rightarrow r)\}$. Hence show that if $p \Rightarrow q$, $p \Rightarrow (q\Rightarrow r) \in Ded(A)$, then $p \Rightarrow r \in Ded(A)$, and so prove the Deduction Theorem without giving an explicit construction for a proof in $Prop(X)$. 
\end{problem}

\begin{flushleft}
\textbf{Lemma}
The deduction of $(p \Rightarrow r)$ goes as follows:\\
$p \Rightarrow (q \Rightarrow r)$                               Given\\
$(p \Rightarrow q) \Rightarrow (p \Rightarrow r)$             Axiom 2\\
$p \Rightarrow q$                                               Given\\
$p \Rightarrow r$                                        Modus Ponens\\
\textbf{End Lemma}
\end{flushleft}

\begin{proof}[Solution 2.6]
Now we give the outlined proof of The Deduction Theorem. Suppose first that $A \vdash p \Rightarrow q$. Adding anything to our assumptions will not remove any deductions. Thus, $A \cup \{p\} \vdash p \Rightarrow q$ remains true. Modus ponens applied to any proof of $p \Rightarrow q$, along with $p$, results in a valid proof of $q$. Therefore, $A \cup \{p\} \vdash q$. 

Suppose now $A \cup \{p\} \vdash q$. We want to show that $A \vdash p \Rightarrow q$. Let n be the length of proof of $p \Rightarrow q$. Suppose that it is true for all $p_i$ with proofs of length $n-1$ or less that $A \cup \{p\} \vdash p_i$ implies that $A \vdash p Rightarrow p_i$. If $q \notin A \cup \mathscr{A}$, (the alternative takes care of the base case of the induction), then the last step in a proof of $q$ from $A \cup \{p\}$ would be modus ponens. Therefore, there exists some $p_i$ deducible from $A \cup \{p\}$ which equals $p_j \Rightarrow q$ where $p_j$ is also deducible from $A \cup \{p\}$, both of which have proofs of length strictly less than $n-1$. 

By the inductive hypothesis, we have $A \vdash p \Rightarrow p_i$ and $A \vdash p \Rightarrow p_j$. Notice that $p \Rightarrow p_i = p \Rightarrow (p_j \Rightarrow q)$. Therefore, we have $p \Rightarrow (p_j \Rightarrow q) \in Ded(A)$ and $p \Rightarrow p_j \in Ded(A)$. By the lemma we have $p\Rightarrow q \in Ded(A)$, which completes the proof of The Deduction Theorem. 
\end{proof}

\begin{problem}[Exercise 2.7]
Show that  $\vdash p\Rightarrow \sim$$\sim$$p$ and construct a proof of $p \Rightarrow \sim$$\sim$$p$ in $Prop(X)$. 
\end{problem}
\begin{proof}[Exercise 2.7]
Recall that $\sim$$p = p\Rightarrow F$. The following is a proof of F from $\{p,\sim$$p\}$. \\Given, $p$\\$\sim$$p = p \Rightarrow F$\\By modus ponens, $F$.\\\\ Therefore,  $\{p,\sim$$p\}\vdash F$. By the Deduction Theorem, $\{p\}\vdash \sim$$p\Rightarrow F$. A second application of the Deduction Theorem gives $\emptyset\vdash p\Rightarrow(\sim$$p\Rightarrow F)$. By definition, we have $\vdash p\Rightarrow\sim$$\sim$$p$. 
\end{proof}
\end{document}
